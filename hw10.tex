\documentclass[12pt]{article}
\usepackage{fullpage,hyperref}\setlength{\parskip}{3mm}\setlength{\parindent}{0mm}
\begin{document}

\begin{center}\bf
Homework 10. Due by 11:59pm on Sunday 11/12.

Linux and the open source software movement.

\end{center}

Linux is the dominant environment for scientific computing. For example, supercomputers generally run some variant of Linux (\url{https://en.wikipedia.org/wiki/Linux}). As another example, most cloud servers are built on Linux, and Linux is therefore dominant for data science applications that involve cloud computing. At University of Michigan, the main resource for high performance computing is the Great Lakes Linux cluster. It should be apparent that Linux skills are useful for a research statistician, as soon as your data analysis or simulation study is too large for a laptop.

Linux expertise in this class ranges from novice to expert. Our goal is to advance our understanding and share knowledge.

Write brief answers to the following questions, by editing the tex file available at \url{https://github.com/ionides/810f23}, and submit the resulting pdf file via Canvas.

\begin{enumerate}

\item Linux, R and Python are all open source and free. 

(a) How do you think these projects led to high quality products given that the  usual financial incentives for building, coordinating and running a development team are missing?

YOUR ANSWER HERE.

(b) If developers are interested in making money, can they do this by writing open-source software? If so, how? If not, why do they do it?

YOUR ANSWER HERE.
  
\item To what extent do you agree or disagree with the opinions at
\url{https://valohai.com/blog/command-line-for-data-science/}?
Specifically,

(i) ``most data scientists are on UNIX-based systems these days.''

(ii) Five reasons why command line use is critical for data science: speed, agnosticism, automation, extensibility, lack of other options.

Unless you vigorously disagree with these arguments, if you have not yet had much experience working with the command line it is time to change that!

YOUR ANSWER HERE.

\item If you are new to Linux, or if your Linux skills are limited to a handful of commands, read the introduction to command line Linux at
  
\url{https://tutorials.ubuntu.com/tutorial/command-line-for-beginners}

If you have a Mac, try out the commands on a Terminal app, which runs a version of Unix that works identically to Linux for many everyday purposes. If you run Windows, try out some commands on the Windows subsystem for Linux

\url{https://docs.microsoft.com/en-us/windows/wsl/}

You can also log in to the UM Linux login service, using SSH Secure Shell. For example from a Mac terminal, type

\texttt{ssh your\_uniqname@login.itd.umich.edu}

Optionally, if you find it an effective way to practice, play the terminus adventure at

\url{https://web.mit.edu/mprat/Public/web/Terminus/Web/main.html}

Report briefly on what you have learned.

YOUR ANSWER HERE.

\item If you are a relatively experienced Linux user, share some words of advice for beginners. How and why did you get started with Linux?  

YOUR ANSWER HERE.

Optionally, you can also use this opportunity to learn some more Linux-related skills, e.g., from the Linux intermediate tutorials at

\url{https://www.linux.org/forums/linux-intermediate-tutorials.124/}


\end{enumerate}
\end{document}
