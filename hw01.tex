\documentclass[12pt]{article}
\usepackage{fullpage,hyperref}\setlength{\parskip}{3mm}\setlength{\parindent}{0mm}
\begin{document}

\begin{center}\bf 
Homework 1. Due by 2:30pm on Monday 8/28. 

What is responsible conduct in research and scholarship (RCRS)? \\
Why are we discussing it? 
\end{center}

Read pages 1--3 of {\em On Being a Scientist}, which is freely available from a link on the course web page, \url{https://ionides.github.io/810f23/}. Edit the tex file, available at \url{https://ionides.github.io/810f23/hw01.tex}, to include the requested answers. Run {\LaTeX} on this file, and submit the resulting pdf file via Canvas.

{\bf You should not have to think long or hard about each question. It is acceptable to write down the first relevant thoughts that come to mind after reading the material. These thoughts will be the starting point for the class discussion. Please write your answers independently, without collaborating with other students. You are not required to consult additional sources, including generative AI programs. If you do, you should cite these sources.

Each answer should be no more than three sentences, and one sentence may be sufficient. If the question is ambiguous, address what you think is the most relevant meaning that the question might have. Be alert for words that have a distinct and non-standard technical meaning in the context of RCRS (``interest'' is a good example).

}

\begin{enumerate}
\item Give one example of how an RCRS issue might arise for each of the following topics, in the context of a Statistics PhD student. We will discuss these topics at greater length later in this course. Here, we are investigating what they mean to you at the start of the course.
\begin{enumerate}
\item Building and maintaining healthy mentor/mentee relationships.

YOUR ANSWER HERE

\item Publication and peer review. 

YOUR ANSWER HERE

\item Data and the reproducibility of research results.

YOUR ANSWER HERE

\item How to avoid mistakes; when a mistake becomes negligence.

YOUR ANSWER HERE

\item Recognizing and responding to conflicts of interest.

YOUR ANSWER HERE

\item Misconduct in research: plagiarism, falsification and fabrication.

YOUR ANSWER HERE

\item Plagiarism while working as a Graduate Student Instructor (GSI).

YOUR ANSWER HERE

\end{enumerate}

\item What are the main sources of funding for research? How has this led to the current requirements on teaching and practicing RCRS?

YOUR ANSWER HERE

\item What role does RCRS play in the value of scientific research and scholarship to society as a whole? What are the benefits and costs to individuals?

YOUR ANSWER HERE

\item How does your reputation as a responsible researcher and scholar become generated and transmitted through the academic community? Optionally, you can consider points on which you agree or disagree with ChatGPT's answer to this question, below.

YOUR ANSWER HERE

\item How important is reputation as a responsible researcher and scholar, in the context of a modern academic career? How does this kind of reputation compare in importance to quantitative measures of academic success (publications in reputable journals; size and number of research grants)? 

YOUR ANSWER HERE

\end{enumerate}

\hrule

{\bf Prompt}

How does your reputation as a responsible researcher and scholar become generated and transmitted through the academic community?

{\bf ChatGPT}

A reputation as a responsible researcher and scholar is built over time through a combination of your actions, contributions, and interactions within the academic community. It involves both your academic work and your behavior in professional and social contexts. Here's how your reputation is generated and transmitted through the academic community:

1. {\bf Quality of Research and Contributions}: Your reputation is primarily established through the quality and significance of your research, publications, and contributions to your field. Producing original, well-researched, and impactful work showcases your expertise and dedication as a scholar.

2. {\bf Ethical Conduct}: Ethical behavior is fundamental to building a positive reputation. This includes proper citation and attribution of sources, adherence to research ethics and guidelines, and responsible data management.

3. {\bf Peer Review and Publishing}: Publishing in reputable journals, conferences, and other academic outlets subjected to rigorous peer review is essential. Successfully passing through peer review signifies the validation of your work by experts in your field.

4. {\bf Collaborations and Networking}: Collaborating with other researchers, both within and outside your institution, can enhance your reputation. Networking helps you share knowledge, learn from others, and establish your presence in the academic community.

5. {\bf Engagement with the Community}: Engaging in academic events such as conferences, workshops, seminars, and panel discussions demonstrates your commitment to the community. Active participation allows you to present your work, exchange ideas, and receive feedback.

6. {\bf Mentorship and Teaching}: Sharing your knowledge and expertise through teaching and mentoring also contributes to your reputation. Effective teaching and mentorship positively influence the academic development of others.

7. {\bf Online Presence}: In the digital age, your online presence matters. Maintaining a professional and informative website, engaging in academic social media, and participating in relevant online discussions can help you reach a broader audience.

8. {\bf Service to the Field}: Volunteering for academic committees, editorial roles, and leadership positions in professional organizations showcases your dedication to advancing your field.

9. {\bf Public Outreach}: Engaging in public outreach activities, such as writing popular science articles, giving public talks, or participating in community events, can help you establish a reputation beyond academia.

10. {\bf Consistency and Longevity}: Building a reputation takes time and consistent effort. Continuously producing high-quality work and maintaining professionalism in all interactions contribute to a lasting positive reputation.

11. {\bf Feedback and Recommendations}: Positive feedback from peers, supervisors, mentors, and collaborators can strengthen your reputation. Recommendations from respected figures in your field carry significant weight.

12. {\bf Avoiding Controversy}: Being mindful of your behavior, communication, and potential controversies is important. Negative incidents can damage your reputation.

Remember that reputation is a delicate and cumulative aspect of your academic career. It takes time to establish, but maintaining it requires ongoing commitment to excellence, ethical conduct, and engagement with the academic community. Reputation is often transmitted through word of mouth, recommendations, citations, and recognition by peers and institutions in the academic world.

{\bf Prompt}

Summarize this in 3 sentences.

{\bf ChatGPT}

A responsible researcher and scholar's reputation is formed by producing high-quality research, adhering to ethical standards, and engaging with the academic community. This reputation is built through peer-reviewed publications, collaborations, networking, and active participation in academic events. Positive recommendations, consistent contributions, and ethical conduct contribute to the transmission of this reputation within the academic community.

\end{document}
