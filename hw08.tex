\documentclass[12pt]{article}
\usepackage{fullpage,hyperref}\setlength{\parskip}{3mm}\setlength{\parindent}{0mm}
\begin{document}

\begin{center}\bf
Homework 8. Due by 11:59pm on Sunday 10/29.

Negligence, mistakes \& how to avoid them

\end{center}
Researchers, like all other humans, make mistakes. Read pages 12--14 of {\em On Being a Scientist}.  Write brief answers to the following questions, by editing the tex file available at \url{https://github.com/ionides/810f23}, and submit the resulting pdf file via Canvas.

\begin{enumerate}

\item How should one balance the professional consequences of errors with the professional requirement to publish?
  
YOUR ANSWER HERE

\item What is a reasonable level of skepticism about correctness of published results?

YOUR ANSWER HERE
  
\item If you think you have identified an error in someone else's published paper, what are possible courses of action? What are their advantages and disadvantages?

YOUR ANSWER HERE

\item Look on the internet in a leading statistics journal (such as Journal of the American Statistical Association and Annals of Statistics) for papers with a corrigendum, erratum or retraction. How common is it in our discipline to publish corrections of mistakes? If you find it is rare, how does the field avoid building on incorrect results?

YOUR ANSWER HERE

\item What kinds of errors arise in statistics research, and what are good research practices to avoid or reduce them?

(a) in theoretical results;

YOUR ANSWER HERE
    
(b) in numerical results.

YOUR ANSWER HERE

\item {\bf A capstone question}. This is the last homework on the topic of responsible conduct in research and scholarship.  Here is a final RCRS question, which concerns all the classes 1--8. We have now discussed various incentives for collegial behavior and consequences for antisocial behavior, in the context of research, teaching and professional service in our field. Can you think of situations where the consequences for RCRS violations are:

  (a) Too lenient. Consequences insufficient to effectively disincentivize antisocial behavior; or too much burden of evidence required to impose consequences; or strong incentives not to impose adequate consequences.

  (b) Too harsh. Disproportionately damaging consequences for minor violations; or penalties imposed with too little burden of evidence; or no presumption of innocence; or incentive structures that lure people into breaking rules and then punish them for it.
  
  
\end{enumerate}

\end{document}
